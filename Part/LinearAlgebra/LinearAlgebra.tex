\documentclass[../../main.tex]{subfiles}

\begin{document}
\chapter{相似变换}\label{相似变换}
相似变换,表征的是线性空间的某一线性映射在坐标变换前后的不同矩阵表示。
因为相似变换是一逆一正的变换,结合张量理论,矩阵是$(1,1)$型张量在特定基底下的分量

一个映射在不同基底下形式不同,最简单的形式肯定是对角矩阵,这样象和原象的元素就是一一对应关系了,由此我们引出相似对角化的概念,即把一个矩阵相似变换成对角矩阵

根据相似对角化理论,一个矩阵相似对角化后的对角阵由特征值构成,相似因子由特征向量构成。进一步,如果矩阵的各特征向量正交,那么相似因子就是一个酉矩阵,相似对角化特殊化为酉对角化
\[
\text{相似对角化}\begin{cases}
    \text{酉对角化}\begin{cases}
        \text{正交对角化} \\
        \text{其他}
    \end{cases}\\
    \text{其他}
\end{cases}
\]
\begin{note}
    \begin{enumerate}
        \item 相似对角化的条件是$k$重特征值有$k$个线性无关的特征向量
        \item 一个矩阵能酉对角化的条件是这个矩阵的互异特征值正交。因为相同特征值值对应的特征向量总能正交化。这种矩阵由正规矩阵表示
        \item 正交对角化是酉对角化的特例,因为正交矩阵是实矩阵,所以矩阵的正交对角化的额外要求是特征向量为实向量(这也是正交矩阵各特征矩阵正交但不一定能正交对角化的原因)
    \end{enumerate}
\end{note}

\section{谱分解定理}\label{谱分解定理}
\begin{theorem}[谱分解定理]
    若$A$能相似对角化,则$A=\sum_i \lambda_i A_i$
    其中$\lambda_i$为矩阵$A$的特征值,$A_i$满足下列三个性质
    \[
        \begin{cases}
            A_i^2=A_i \\
            A_iA_j=0\;(i\neq j)\\
            \sum_i A_i=I
        \end{cases}
    \]
\end{theorem}
\begin{enumerate}
    \item 上述定理的证明借助相似对角化的定义,把$A$用对角阵和相似因子表示出来,再把对角矩阵分解就能得证,$A_i$的性质借助$A_i$的定义不难得出
    \[
    A=P\Lambda P^{-1}=\sum_i \lambda_i P E_i P^{-1}=\sum_i \lambda_i A_i
    \]
    \item 注意这里的$E_i$并不是单位阵,而是对角线上第$i$个元素为$1$且其余元素全为$0$的矩阵。
    \item 把$P$矩阵的各列表示成$\mathbf{v}_i$,把$P^{-1}$矩阵的各行表示成$\mathbf{\omega}_i^T$,那么有
    \[
    A_i = P E_i P^{-1} = \begin{bmatrix}
        0\cdots\mathbf{v}_i\cdots 0
    \end{bmatrix}\begin{bmatrix}
        \mathbf{\omega}_1^T \\ \vdots \\ \mathbf{\omega}_i^T \\ \vdots \\ \mathbf{\omega}_n^T
    \end{bmatrix}=\mathbf{v}_i\mathrm{\omega}_i^T
    \]
    \begin{note}
    如果$A$可以酉相似化,不难证明$A_i=\mathbf{v}_i\mathbf{v}_i^{\dagger}$
    
    在量子力学中则表示为$A=\sum_i \lambda_i \ket{\alpha_i}\bra{\alpha_i}$,其中$\ket{\alpha_i}$是对应于特征值$\lambda_i$的本征矢。
    \end{note}
    \item 借助谱分解定理,可以方便的计算矩阵的函数,我们这里关注厄米矩阵
    \[\begin{split}
        f(A)& =\sum_i f(\lambda_i \ket{\alpha_i}\bra{\alpha_i})\\
        &=\sum_i f(P\lambda_i \Lambda P^{-1})\\
        &=\sum_i Pf(\lambda_i \Lambda )P^{-1}\\
        &=\sum_i f(\lambda_i)P \Lambda P^{-1}= \sum_i f(\lambda_i) \ket{\alpha_i}\bra{\alpha_i}
    \end{split}
    \]
    \begin{note}
        这里的$\Lambda$是第一个元素为$1$,其余元素为$0$的对角矩阵。因为$\ket{\alpha_i}\bra{\alpha_i}$的特征值是$1$和$0$
    \end{note}
\end{enumerate}

\chapter{特征值与特征向量}\label{特征值与特征向量}
\begin{enumerate}
    \item $n$维矩阵在复数域上有$n$个特征值,这是代数基本定理保证的
    \item 考虑简并,$k$重特征值至多有$k$个线性无关的特征向量
    \item 一个特征值对应的特征向量构成一个线性空间(同一特征值的特征向量的线性组合仍是这一特征值的特征向量),称为本征子空间(其实就是特征方程的零空间),它的维数(线性无关的特征向量的个数,即线性无关的解的个数)由这个特征值对应的特征矩阵的秩决定,即$d=n-r$
\end{enumerate}

\chapter{复线性空间}\label{复线性空间}
\begin{enumerate}
    \item 实线性空间与复线性空间的区别,根源来自内积的定义不同。前者内积是分量的平方和,后者是分量的模的平方和。这种差异是为了保证复向量的内积是实数(因为向量与自身的内积是向量的长度的平方)(当然这里的内积是定义在正交基底下的)
    \item 将上述差异写成表达式,就是把转置换成共轭转置,对于不涉及正交的定理,自然不会改变(比如矩阵相似对角化的条件),而对于涉及正交的定理,也不是简单的直接推广,不明确最好不要瞎用
    \item 作为对应,复线性空间的共轭转置性质和实线性空间的转置性质类似。当然复线性空间也有转置的概念,性质和实线性空间一样
    \item 转置,共轭转置,取逆三个操作顺序可以互换
    \item 实对称阵对应厄米矩阵,正交阵对应酉矩阵
    \begin{note}
        在\ref{正规矩阵}小节,我们会看到,从对角化的角度,实对称阵是一个极特殊的矩阵,不能完全说对应厄米矩阵
    \end{note}
    \end{enumerate}

\section{正规矩阵}\label{正规矩阵}
\begin{definition}
    矩阵$A$称为正规矩阵,若$A$满足$AA^\dagger=A^\dagger A$
\end{definition}
由此定义可知:正规矩阵和它的伴随矩阵是可交换的,我们可以简单对正规矩阵做个分类
\[
    \text{正规矩阵}\begin{cases}
        \text{厄米矩阵}\begin{cases}
            \text{实对称阵} \\
            \text{其他}
        \end{cases} \\
        \text{酉矩阵}\begin{cases}
            \text{正交阵} \\
            \text{其他}
        \end{cases} \\
        \text{其他}
    \end{cases}
\]

正规矩阵有三个重要性质,如下所示[证明由deepseek提供]
\begin{theorem}
    如果 \( A \) 是正规矩阵,且 \( \mathbf{v} \) 是 \( A \) 的对应于特征值 \( \lambda \) 的特征向量,则 \( \mathbf{v} \) 也是 \( A^\dagger \) 的对应于特征值 \( \overline{\lambda} \) 的特征向量
\end{theorem}
\begin{proof}
    首先考虑矩阵 \( B = A - \lambda I \).
   \[
   B^\dagger B = (A^\dagger - \overline{\lambda} I)(A - \lambda I) = A^\dagger A - \overline{\lambda} A - \lambda A^\dagger + |\lambda|^2 I
   \]
   \[
   B B^\dagger = (A - \lambda I)(A^\dagger - \overline{\lambda} I) = A A^\dagger - \overline{\lambda} A - \lambda A^\dagger + |\lambda|^2 I
   \]

   由于 \( A^\dagger A = A A^\dagger \),所以 \( B^\dagger B = B B^\dagger \),即 \( B \) 是正规矩阵

    接着对于任意向量 \( \mathbf{x} \),有:
   \[
   \|B \mathbf{x}\|^2 = \langle B \mathbf{x}, B \mathbf{x} \rangle = \mathbf{x}^\dagger B^\dagger B \mathbf{x}
   \]
   \[
   \|B^\dagger \mathbf{x}\|^2 = \langle B^\dagger \mathbf{x}, B^\dagger \mathbf{x} \rangle = \mathbf{x}^\dagger B B^\dagger \mathbf{x}
   \]

   由于 \( B^\dagger B = B B^\dagger \),所以 \( \|B \mathbf{x}\| = \|B^\dagger \mathbf{x}\| \) 对所有 \( \mathbf{x} \) 成立

    最后特别地,对于特征向量 \( \mathbf{v} \)(满足 \( A \mathbf{v} = \lambda \mathbf{v} \),即 \( B \mathbf{v} = 0 \)):
   \[
   \|B^\dagger \mathbf{v}\| = \|B \mathbf{v}\| = 0
   \]

   这意味着 \( B^\dagger \mathbf{v} = 0 \),即:
   \[
   (A^\dagger - \overline{\lambda} I) \mathbf{v} = 0 \quad \Rightarrow \quad A^\dagger \mathbf{v} = \overline{\lambda} \mathbf{v}
   \]
\end{proof}
\begin{theorem}
    如果 \( A \) 是正规矩阵,且 \( \lambda_1 \) 和 \( \lambda_2 \) 是 \( A \) 的两个不同的特征值,对应的特征向量分别为 \( \mathbf{v}_1 \) 和 \( \mathbf{v}_2 \),则 \( \mathbf{v}_1 \) 和 \( \mathbf{v}_2 \) 正交
\end{theorem}

\begin{proof}
    易知:
    \[
    A \mathbf{v}_1 = \lambda_1 \mathbf{v}_1, \quad A \mathbf{v}_2 = \lambda_2 \mathbf{v}_2
    \]
    \[
    A^\dagger \mathbf{v}_1 = \overline{\lambda_1} \mathbf{v}_1, \quad A^\dagger \mathbf{v}_2 = \overline{\lambda_2} \mathbf{v}_2
    \]

    考虑内积 \( \langle A \mathbf{v}_1, \mathbf{v}_2 \rangle \):
    一方面:
     \[
     \langle A \mathbf{v}_1, \mathbf{v}_2 \rangle = \langle \lambda_1 \mathbf{v}_1, \mathbf{v}_2 \rangle = \overline{\lambda_1} \langle \mathbf{v}_1, \mathbf{v}_2 \rangle
     \]

    另一方面,利用内积的性质 \( \langle A \mathbf{u}, \mathbf{v} \rangle = \langle \mathbf{u}, A^\dagger \mathbf{v} \rangle \):
     \[
     \langle A \mathbf{v}_1, \mathbf{v}_2 \rangle = \langle \mathbf{v}_1, A^\dagger \mathbf{v}_2 \rangle = \langle \mathbf{v}_1, \overline{\lambda_2} \mathbf{v}_2 \rangle = \overline{\lambda_2} \langle \mathbf{v}_1, \mathbf{v}_2 \rangle
     \]
  
    将两者相等:
    \[
    \overline{\lambda_1} \langle \mathbf{v}_1, \mathbf{v}_2 \rangle = \overline{\lambda_2} \langle \mathbf{v}_1, \mathbf{v}_2 \rangle
    \]
    \[
    (\overline{\lambda_1} - \overline{\lambda_2}) \langle \mathbf{v}_1, \mathbf{v}_2 \rangle = 0
    \]

    由于 \( \lambda_1 \neq \lambda_2 \),所以 \( \overline{\lambda_1} - \overline{\lambda_2} \neq 0 \),因此
    \[
    \langle \mathbf{v}_1, \mathbf{v}_2 \rangle = 0
    \]
    即 \( \mathbf{v}_1 \) 和 \( \mathbf{v}_2 \) 正交
\end{proof}

\begin{theorem}
    矩阵 \( A \) 是正规矩阵当且仅当它能酉相似对角化
\end{theorem}
这个定理不作证明

由上述定理,我们立马能想到,在实线性空间,实正规矩阵的互异特征值对应的特征矢量正交,那理应能推导出实正规矩阵能正交对角化,但显然不是(正交矩阵就是实正规矩阵,但不一定能正交对角化)。关键点我们已在前文提到,实正规矩阵的特征向量不一定是实向量,那它的相似因子也不一定是实矩阵,但正交矩阵要求是实矩阵,所以实正规矩阵不一定能正交对角化,正能退而求其次,一定能酉对角化

实对称阵特殊在所有特征值都是实数,自然特征向量也是实向量,相似因子就能是正交矩阵了。从这个角度,实对称阵和厄米矩阵是不对等的(两者对等意味着正交矩阵能正交对角化,但显然不一定)

\section{厄米矩阵}\label{厄米矩阵}
\begin{enumerate}
    \item 实线性空间中对称矩阵性质特殊,它满足$A=A^{\mathrm{T}}$,进而可推出它的互异特征值对应的特征向量正交。在复线性空间中,自然可以定义满足$A=A^\dagger$的矩阵,显然它的互异特征值对应的特征向量也正交。这种矩阵称为厄米矩阵(根据上文的\ref{正规矩阵}小节,显然厄米矩阵只是这类矩阵的子集)
    \item 厄米矩阵的本征值均为实数
    \item 厄米矩阵的一组线性无关的本征矢能构成线性空间的基(因为它能相似对角化),进过正交化处理,厄米矩阵的本征矢能构成一组正交基
    \item 厄米矩阵能通过酉变换进行对角化,根据\ref{谱分解定理}我们有$A=\sum \lambda_i \ket{\alpha_i}\bra{\alpha_i}$
\end{enumerate}

\section{酉矩阵}\label{酉矩阵}
\begin{enumerate}
    \item 正交阵和酉矩阵的特征值的模是都为$1$,特征向量的模都为$1$,两者的行列式的绝对值都为$1$
    \item 正交阵和酉矩阵都保内积不变。其实这里的保不变是强制内积为正交基下的形式(就是上面的定义)后,再进行坐标变换,讨论内积不变。在张量分析视角下,内积由度规和分量决定,是标量,坐标变换下自然是一定不变的,而正交阵和酉矩阵的特殊之处在于保证了度规形式不变(相对于正交基不变)
\end{enumerate}
\end{document}