\documentclass[../../main.tex]{subfiles}
\begin{document}
微分几何是一门借助微积分分析复杂几何的内在特性的学科,作为一门数学学科,有高度的抽象性,下文书写的相关内容,基本都不是数学上的定义,都是形象化的阐述或者针对物理人比较具象的定义,与严谨的数学定义可能有很大出入(因此也不一定对)

微分几何的研究对象自然是几何,但不是毫无特点的几何,我们称之为流形。所谓的流形,通俗来说是一片连续光滑的几何,能用一组连续坐标标定几何中的点。欧几里得空间是平凡的流形,一般的流形整体上是弯曲的,但局部可以近似为欧几里得几何,且整个流形可以嵌在更高维的欧几里得空间

\begin{note}
    并不是所有的流形都能被想象出来,比如闵可夫斯基时空,是四维流形,配有闵氏度规,很难想象出来这个几何长什么样,但不影响运算,这就是微分几何的优势(后面会看到,给出度规,流形就完全确定了)
\end{note}
\begin{example}
    常见的流形比如三维空间里的连续曲面,是二维流形,比如球面,锥面等。
\end{example}
借助流形所嵌入的高维欧几里得空间研究流形,是古典微积分的研究范围,比如计算曲线的曲率(曲线是一维流形,这里的曲率是外曲率)。这种研究偏向研究几何的外在属性,而不是几何本身自带的,研究后者的学科就是微分几何

为了研究流形内在的性质,又考虑到流形能用一组坐标表示,显然这种坐标表示是不唯一的,而几何的内在属性应该和选取的坐标无关,所以我们要研究一种坐标变换下不变的量,来表征几何的各种内在属性。这种量我们称之为张量

\begin{note}
    物理中也需要这种量,根据相对性原理,物理规律应该有坐标变换不变性,而物理规律是用物理量的方程表示的,所以物理量要坐标变换不变
\end{note}
一个流形除了能被一组连续的坐标标定之外,没有任何其余的额外通用性质,比如流形上两点之间的距离,是没有定义的,某种意义上可以任意定义。我们先从欧几里得空间入手,分析它的张量场和各种性质,再把这些性质抽象推广到一般流形

\begin{note}
    欧式空间的优点是平直且有一个自然的距离定义。平直意味着平移一个矢量后它依然是矢量,自然的距离定义即勾股定理,能导出几何最重要的性质:度规
\end{note}


\chapter{欧式空间的张量}
这里讨论欧氏空间,肯定不能着眼于直角坐标,而是着眼于任何连续的坐标。直角坐标只适用于平直的流形,而任意的连续坐标,便于向弯曲的一般流形推广

\begin{example}
    以二维欧几里得空间即平面为例,常见的坐标是\((x,y)\)或者\((r,\theta)\)当然也可以是任意坐标,比如\((p,q)\)其中\(p=x;q=y^3\)
\end{example}
给定欧氏几何的一组坐标后,一切应当从基矢谈起,而在一般的流形上,我们要抛弃这个概念。定义基矢后,我们再引入对偶基和度规,接着引入张量,最后借助张量的坐标变换不变性导出张量分量的坐标变换式,最后我们引入协变导数表示张量场对坐标的变化,这是为一般流形做准备的
\begin{definition}
    设\(n\)维欧氏空间由坐标\(\{x^i\}\)表示,则\[\mathbf{g}_i=\frac{\partial\mathbf{r}}{\partial x^i}\quad i=1,2,\ldots,n \]为\((x^1,x^2,\ldots,x^n)\)点的基矢
\end{definition}
这里坐标的上角标显然不是次幂的意思,只是不同坐标的标号,放在上侧,显然是历史原因(因为上下是相对的),但后面会看到,上下标的含义是不同的。同样基矢的下标也不是随意放置的,我们称这种指标在下方的基矢为协变基矢
\begin{note}
    有个口诀是上逆下协
\end{note}

显然基矢是逐点定义的,因为位矢并不是随坐标线性变化的,在每个点都有\(n\)个基矢。基矢的几何含义是位矢在某个点延坐标线增大方向的单位变化

接下来基于协变基矢构造矢量
\begin{definition}
    设\(\mathbf{g}_i\)是\(p\)点的基矢,定义
    \[\mathbf{v}=v^i\mathbf{g}_i\]为\(p\)的的矢量
\end{definition}
上述协变基矢和矢量的定义也适用于一般的流形,显然\(p\)点的协变基矢与流形是相切的所有矢量构成了一个线性空间,称之为切空间,而\(p\)点与它的切空间组成的有序对构成的集合成为流形的切丛

因为基矢是逐点定义的,所以切空间也是逐点定义的,不同点的矢量是无法用分量比较的,因为属于不同的空间

现在着眼于某个点,它的基矢可能性质非常差,既不正交,也不归一。计算切空间中两个矢量的内积非常不方便(实际上我们都没有定义内积,这里沿用欧氏空间的定义,计算之所以不方便是不同基矢不正交,内积不为\(0\)),为了让内积计算方便,
我们在切空间中找另一组基



借助基矢,我们可以定义一个更为抽象的量,张量
\begin{definition}
    
\end{definition}
\chapter{一般流形上的张量}
这里的位矢是站在更高一维的欧氏空间下定义的,针对流行本身,并不存在位矢,后面也会看到,这样的位矢一般也不是流形上的矢量。
\end{document}