\documentclass[../../main.tex]{subfiles}

\begin{document}
从客观的课程物理难度上说,量子力学无疑是物理学院本科专业课最难的一门。抽象或者更恰当的说是陌生的物理图像,加上繁琐的计算,还有部分学生对其如“圣经”般的态度\footnote{在我认为,经典力学更是圣经中的圣经}给这门科目带来了许多学习障碍。量子力学和大家熟悉的经典力学图像截然不同,所以我认为在学习之前特别有必要先澄清这些不同,给大家带来观念上的转变,至于具体的理论细节,大家可以在学习中自行感悟。
\section{向量}
我们首先要阐述一些数学\footnote{当然大家数学水准各不相同,这节主要关注数学,熟悉的人可以直接跳过}上的概念。向量,这个物理学的核心数学工具,是观念转变的第一步

\subsection{数学在干什么}
在展开具体讨论之前,有必要简单讨论一下数学本身的特点,毕竟物理学中使用的数学犹如工科中使用的物理,很多人除了计算没有别的观感了。比起大家课上学习的各种数学定理和计算公式,数学真正抽象的地方在于数学只定义性质而不限制内容。

以现在的主题向量为例,数学家口中的向量只是线性空间中的元素,至于线性空间,是一个元素满足一些特殊性质的集合,至于这个集合具体是什么,数学家并不在意。大家很熟悉的有大小有方向的量或者说用一组数字表示的向量\footnote{当然这种向量有着特殊意义}只是数学上向量的一个例子,是一个具体的线性空间。另一个大家熟悉的概念是积分。物理人提到积分,第一反应就是黎曼和,这跟物理人面临的物理情景中的微元求和思想一致,这也是牛顿老爷子创立微积分的初衷。但数学上的积分却不限于此,著名的勒贝格积分就是只限制性质的积分,对于不同的问题,我们要构造一个恰当的勒贝格积分进行计算。当然数学肯定也不是无根之木,线性空间是从箭头这类具体的向量中抽象出来的概念,勒贝格积分也是在黎曼积分的基础上发展出来的。

大家常说数学的抽象在于公理化体系,从少数公理出发推导大量的命题。“公理化”这个词本身就略带抽象,在我看来,可以理解为只定义某个概念的性质,而不限制具体的内容,由此人们可以拿来构造或者从自然发现各种的实例\footnote{有点类似编程概念里的类与对象},直接使用数学家给出的结论。

数学的这个特点至少能够我们两点启发:一是遇到和自己所学不一致甚至相悖的新数学知识时不要惊讶,二是在讨论一个数学概念的时候一定要明确其当前的确切定义(显然这一点也适用于物理概念的讨论)。

\subsection{向量作为基的线性组合}
正如上一小节所言,我们很难说向量是什么,因为很难定义线性空间是什么。但经过下面的论述,大家会体会到向量一个抽象但格外具体的含义。

线性代数告诉我们,在一个线性空间中可以找到无数组向量组,每个向量组里的各向量线性无关且空间中的每个向量都可以由这个向量组唯一线性表示。这样的一组向量叫线性空间的基底,简称基,基底中向量的个数称为线性空间的维数。下面我们把这些概念的逻辑倒过来说一遍。

我们先找几个\footnote{有限个或无穷个}元素,接下来定义这些元素之间的加法以及这些元素和实数(或者复数)的乘法,最后计算出实数和这些元素的所有线性组合,把得到的新元素放在一个集合里,显然这个集合是一个线性空间,空间的维度是最开始选的元素的个数。

上述过程被称为张成(英文为span),指导我们如何凭空构造一个线性空间。因为对最开始选择的几个元素没有任何限制,所以这样的线性空间是高度抽象的,但里面的元素即向量又很具体,仅仅是几个基底的线性组合。我们用这种观点理解两个线性空间。

第一个线性空间是\(\mathbb{R}^3\),这个空间大家很熟悉,选三个互相垂直的单位长度箭头\(\{\hat{\mathbf{\mathrm{e}}}_x,\hat{\mathbf{\mathrm{e}}}_y,\hat{\mathbf{\mathrm{e}}}_z\}\),它们的线性组合给出了三维实空间。这个空间没有很好体现张成的概念,因为\(\mathbb{R}^3\)不用构造就客观存在。

第二个线性空间很难说具体是哪个空间。我们选择\(\{x^n\}_{n=0}^{\infty}\)为基底\footnote{这个集合中的\(n\)取遍所有的非负整数,\(x\in \mathbb{R}\)},用它们的线性组合构造一个线性空间。我们对这个空间简单讨论一下,显然这个空间是无穷维的,因为有无穷个基。这个空间里的元素都是函数,因为基底都是函数。\(\mathrm{e}^x\)属于这个空间,因为\(\displaystyle \mathrm{e}^x=\sum_{n=0}^{\infty}\frac{x^n}{n!}\)是基底的线性组合。因为基底是连续函数,我们讨论一下空间中元素的连续性。这个空间中的函数一定连续吗?一定连续,这是由幂级数的性质决定的。反过来问,连续函数一定在这个空间吗?不一定,比如\(|x|\)就不能表示成幂函数的线性组合\footnote{函数在原点不是无穷阶可导的都不在这个空间内}。\(\mathbb{R}^3\)中的元素有大小的概念,这个空间中的元素有大小吗?貌似函数没有几何意义上的大小,但我们可以定义一个大小,进一步,我们可以定义两个函数的距离,这都是后话里了。回到本段第一句话,这样的一个线性空间是否能像\(\mathbb{R}^3\)一样给出一个确切地名字以表示其中的元素,可能可以,似乎是一元无穷阶可导函数空间,但这有赖于严格的数学证明。无论如何,这个空间一定是线性空间,线性空间的结论可以直接套用。

从这两个例子可以看出,把向量看作基的线性组合是一种具体而不失抽象的认识。本小节的主要目的就是在这种观点之上让大家接受函数也可以是向量,从线性空间的角度看,函数和立体几何中的一个箭头没什么区别。关于函数可看成向量,还有一种感性的认识。

考虑一个\(N\)维向量\(V\),\(V^\mu\)是向量\(V\)的第\(\mu\)个分量,也可以了理解为向量\(V\)在\(\mu\)这个位置的分量\footnote{准确来说是将\(V\)以当前基底线性展开,第\(\mu\)个基的系数},\(\mu\)这个角码标明了第几个位置。我们借此来思考函数\(f(x)\),那么我们可以认为它表示向量\(f\)在\(x_0\)这个位置的分量,分量值是\(f(x_0)\),\(x_0\)标明了分量的位置,不过这里的位置不再是正整数,而是遍历整个实数,但不妨碍我们仍然称之为“位置”。同样对于三元函数\(f(\mathbf{r})\),我们可以理解它为向量\(f\)在\(\mathbf{r}_0\)这个位置的分量,分量值是\(f(\mathbf{r}_0)\),\(\mathbf{r}_0\)遍历整个三维空间。这里有个值得注意的点是\(\mathbf{r}_0\)本身是三维空间中的向量,它是三个基的线性组合,同时它又对应了函数\(f\)所在的线性空间的一个基:也就是说,三个正交归一的箭头张成了一个线性空间,这个空间的每个箭头又对应一个函数,这些函数又张成了一个线性空间。具体内容我们到正文部分详细说明。

\subsection{数域——为什么是复数}
前文提到,张成一个线性空间的时候,我们可以定义基底和实数的乘积也可以定义和复数的乘积,两者的区别是线性空间的数域不同,前者是实数域,后者是复数域,进而我们称前者为实线性空间,后者为复线性空间。在没有学习量子力学之前,大家接触的线性空间,无论是线性代数课程还是力学、电磁学等普通物理课程,基本都是实线性空间,甚至大家日常使用的数字也是实数居多。电磁学和光学会引入复振动,借复数来表示振动,但教材通常会强调这只是数学上的方便,便于叠加求导等等。光学提及的复折射率,并不是指折射率是个复数,其中的虚数部分反应光在介质中的衰减。也就是说,在量子力学之前的课程,复数的使用仅仅是为了方便,完全可以不使用,但量子力学不是这样的,量子力学建立在一个复线性空间上,量子力学研究的对象生活在复数的世界里,这里的复数是必须的,不使用复数\footnote{严格来说,是不使用复结构,在各种复结构的量子力学版本中,复数版本的量子力学最简洁}得不到现在如此自洽且应用广泛的量子力学。

线性代数课程中,一个实矩阵的特征值可能是复数,这其实在暗示在复数这个数域下讨论线性空间更合理。量子力学和数学为什么选择了一个大自然不存在\footnote{细想一下自然界貌似也不存在负数}的虚数作为理论的基石,为什么这个数域下的理论如此自洽优美,目前为止应该没有完善的解释。随着大家学习的深入,可能会看到各种论述来阐明量子力学使用复数的必要性,但就我的学习经历来看,大部分论述都是从不同角度阐述量子力学不使用复数不可以,而没有直接论述为什么要使用复数,在这里提出这个问题,无意让大家深思,毕竟作为学习者,接受也是学习的一部分。

当然,就理论本身而说,复线性空间绝不只是涉及的数字从实数变到复数这么简单,复线性空间扩展了许多实线性空间的概念,许多结论也要进行修改推广,在学习过程要注意不要把线性代数课程上学习的实线性空间性质一成不变的套用到量子力学中。

\subsection{无穷带来了什么}
\subsubsection{什么是无穷维线性空间}
 我们首先要讨论一下什么是无穷。高等数学用\(\varepsilon-\delta\)语言严格的定义了极限,接着定义了两种特殊的极限:一种非负但比任意正数都小,即无穷小;一种绝对值比任意正数大,即无穷大,简称无穷。无穷大视极限值的正负又可分为正无穷和负无穷,显然,在本小节的语境下,无穷指的是正无穷。

 之所以要明确这个概念,是为了让大家认识到无穷不是一个数字,它是作为极限存在的,因此去比较两个无穷的大小没有意义。高中初次接触复数,教材就表明复数不可比较大小,因为它们不是实数,无穷大甚至不是一个数,比较其大小更是无稽之谈。所以所谓\(\infty\text{和}\infty+1\)哪个大这种问题毫无意义\footnote{甚至\(\infty+1\)这个式子本身就没有意义}。

 至于什么是无穷维线性空间,根据前文张成的概念,很容易理解无穷维线性空间即无穷个基底张成的空间。因为无穷不是数字,所以谈论无穷维线性空间的维度也是没有意义的。
\subsubsection{可数无穷和不可数无穷}
无穷不可比较大小,那么两个无穷大集合就不能比较元素个数的多少,对此有一个很有意思的论述:希尔伯特旅店问题。一个“注满”客人的无穷间客房的酒店仍然可以再住进无穷个客人。但是朴素的想法是,自然数比偶数多,实数比自然数多。为了让这种朴素的想法严格化,数学家发明了一个新的概念来表示无穷的大小:势。这个概念是如此的简单以至于非常的合理:两个无穷大的集合,如果两个集合中的元素可以建立一个双射,则两个集合等势,在势这个意义上,两个集合元素一样多。

这是一个很自然的定义,因为双射即一一映射,两个无穷大的集合中的元素能一一对应,自然两个集合等大。但这个自然的定义会导出一些奇怪的结论。比如驳斥我们朴素观念的:自然数和偶数一样多,因为\(n=2l\left(l \in \mathbb{N}\right)\)这个双射建立了自然数和偶数的一一映射,显然自然数和奇数也一样多。但这太奇怪了,很显然自然数是包含偶数的,这种诡异来自无穷,进入无穷的世界,没有人知道会发生什么。我们无法找到自然数和实数之间的双射,这样两者就不等势了,至于谁的势大,显然实数的势大,所以我们可以说实数比自然数多。但是考虑\(y=\arctan(x)\)这一双射,我们又会得到一个奇怪的结论,即实数集和它的一个子集\(\left(-\frac{\pi}{2}\text{,}\frac{\pi}{2}\right)\)等大。我们还可以讨论一下线性空间的势,对于有限维空间,比如\(\mathbb{R}^3\)可以证明它和它的子集实数集是等势的。至于无穷维线性空间,它的势应该是大于实数集的势的,这有赖于严格的数学证明。

至此我们讨论了两个典型的无穷大集合,自然数集和实数集。对于和自然数等势的无穷集合,我们称它的元素数为可数无穷;对于和实数等势的无穷集合,我们称它的元素数为不可数无穷。学习中接触的无穷集合大多都是这两种,比实数集势还大的集合也有,比如实数集的幂集。注意所谓的可数和不可数是站在自然数离散而实数连续这个角度命名的\footnote{更不是英语语法中的可数和不可数,不要学串台了},要是有人认为自然数“可数”,那他会很不幸,因为他花一辈子也不会数尽自然数。
\subsubsection{鬼知道无穷带来了什么}
前文已经提到:进入无穷的世界,没有人知道会发生什么。很多结论开始变得反常。量子力学使用的数学空间是希尔伯特空间,即完备的内积空间。内积空间指定义了内积的线性空间,内积大家有所了解,而完备却是陌生概念。这是个并不复杂的概念,但在前言部分不适合引入过多新概念,简单来说,一个完备的线性空间\footnote{本小节提及的线性空间都是定义了内积的,因为我们主要讨论完备}就是说这个空间中的收敛序列,收敛到这个空间的元素,不会收敛到空间之外。

一个空间中的收敛序列,怎么会收敛到空间外的一个元素上呢?高等数学中学习的一种收敛序列即收敛数列,一定会收敛到一个实数,不可能一个收敛数列收敛于一个二维向量,这很不可思议!对数列来说,确实不会这样,因为收敛数列是\(\mathbb{R}^1\)中的一个收敛序列,\(\mathbb{R}^1\)是一维线性空间,有限维线性空间总是完备的,收敛序列收敛于空间内的元素。但对于无穷维线性空间,这不一定,一个收敛序列可能会收敛到空间之外,这会导致一个不完备的线性空间,这种例子很多,不难在互联网上获得。不完备的空间显然是不好的空间,现代数理理论建立在极限上,不完备空间会出现极限值不在空间中,带来后续分析的困难,因此量子力学采用的是完备空间:希尔伯特空间。线性空间维数到达无穷会出现性质不好的空间,剔除这些不好的空间,剩下的就是希尔伯特空间\footnote{有限维线性空间显然是希尔伯特空间,但一般不参与讨论}。

依据前文对无穷的划分,如果张成希尔伯特空间的基底是可数无穷个,那么这个希尔伯特空间称为可分希尔伯特空间;如果张成希尔伯特空间的基底是不可数无穷个,那么这个希尔伯特空间称为不可分希尔伯特空间。有意思的是自然数是实数的子集,在特定情况下,可分希尔伯特空间也会是不可分希尔伯特空间的子空间,这个特点有助于后续大家理清右矢和广义右矢的关系。

除了线性空间不再完备,无穷还会带来很对例外,大家熟悉的比如无穷级数和积分的换序问题。对有有限求和,积分和求和是任意换序的,但对于无穷级数,换序是需要条件的,物理学家往往直接运算而把证明留给数学家,但科学史上确实存在无穷级数和积分错误换序导致乌龙的“趣事”。要完整的列出无穷相对于有限带来了什么恐怕是不可能的,因为我怀疑带来的反常本身就无穷多,或许,真的只有鬼才知道无穷带来了什么。

\section{状态}


\section{物理量}
\section{测量}

\end{document}